\section{Описание методов и классов}

В данном разделе приведены сведения об основных классах и методах серверной части
программного средства.

\begin{center}
\begin{longtable}{ 
      | >{\centering}m{0.28\textwidth} 
      | >{\centering}m{0.31\textwidth} 
      | >{\centering\arraybackslash}m{0.35\textwidth}|}
  \caption{Основные классы и методы пакета <<service>>}
  \label{table:service_reference}\\
  \hline Класс & Метод  & Описание\\ \hline
  \endfirsthead
  \caption*{Продолжение таблицы \ref{table:service_reference}}\\
  \hline 
    Класс & Метод & Описание\\
  \hline
  \endhead
  \hline
  \endfoot
  	\multirow{6}{*}{AbstractCrudService} 
  	& checkValidNestedEntities-\\IfNeed(entity: E) 
  	& Метод, предназначенный для валидации вложенных сущностей, где они передаются в качестве 
  	параметров метода, по умолчанию пустой, нужен для переопределения в наследниках, где это 
  	нужно \mbox{ServiceNotFoundException}.\\
  \cline{2-3}
  	& checkPersonalVisibility(\\userId: Long, \\entityId: Long) 
  	& Проверяет соответствие принадлежности сущности с entityId пользователю с userId. Если 
  	сущность не принадлежит пользователю, бросает \mbox{ServiceNotFoundException}.\\
  \cline{2-3}
  	& findByIdAndUserId(\\id: Long, \\userId: Long): E?
  	& Находит в базе данных сущность с данными id и userId или возвращает null если сущности с таким
  	 сочетанием признаков не существует.\\
  \cline{2-3}
  	& create(entity: E): E?
  	& Создаёт сущность, переданную в параметре (если она валидна), если она успешно создана, возвращает эту 
  	сущность, с заполненными полями, которые были сгенерированы в процессе создания (например, id). Если произошла 
  	ошибка в процессе изменения, то бросает \mbox{ServiceException}.\\
  \cline{2-3}
  	& update(entity: E): E?
  	& Изменяет сущность, переданную в параметре (если она валидна), если она успешно изменена, возвращает эту 
  	сущность, с обновлёнными полями. Если произошла ошибка в процессе изменения, то бросает \mbox{ServiceException}.\\
  \cline{2-3}
  	& delete(id: Long, \\userId: Long): Unit?
  	& Удаляет сущность по переданным признакам, ничего не возвращает, если операция была 
  	проведена успешно (Unit), возвращает null, если такой сущности не существует. Если 
  	произошла ошибка в процессе удаления, то бросает \mbox{ServiceException}.\\
  \hline
  	\multirow{4}{*}{CategoryServiceImpl} 
  	& delete(id: Long, userId: Long): Unit?
  	& Удаляет категорию по переданным признакам и удаляет все связанные с ней транзакции, 
  	ничего не возвращает, если операция была проведена успешно (Unit), возвращает null, 
  	если такой сущности не существует. Если произошла ошибка в процессе удаления, то 
  	бросает \mbox{ServiceException}.\\
  \cline{2-3}
  	& softDelete(id: Long, \\newId: Long,\\userId: Long): Unit?
  	& Удаляет категорию по переданным признакам и перемещает все связанные с ней транзакции на 
  	категорию с newId. Категория с newId должна быть того же типа, что и удаляемая категория. 
  	Ничего не возвращает, если операция была проведена успешно (Unit), возвращает null, если такой 
  	сущности не существует. Если произошла ошибка в процессе удаления, то бросает \mbox{ServiceException}.\\
  \cline{2-3}
  	& calculateCategoryStats(\\userId: Long, \\type: CategoryType, \\currencyCode: String, \\range: DateRange?\\): CategoriesStats
  	& Возвращает объект статистики, содержщий список всех категорий для данных пользователя и типа, 
  	содержащий статистику по заданной валюте. Если параметр range равен null, тогда статистика подсчитвается 
  	за всё доступное время. \\
  \cline{2-3}
  	& findByUserIdAndType(\\id: Long, \\type: CategoryType\\): List<Category>
  	& Возвращает список всех категории указанного типа type, принадлежащих пользователю с указаным id.\\
  \hline
  	CurrencyServiceImpl
  	& findByCode(code: String)\\: Currency?
  	& Ищет в базе данных валюту с указанным кодом code, если такой валюты не существует --- возвращает null.\\
  \hline
  	TransactionServiceImpl
  	& findAllByUserId(\\userId: Long, \\pageable: Pageable\\): Page<Transaction>
  	& Возвращает страницу транзакций, содержащий список транзакций данного пользователя, количество элеметнов 
  	списка, общее количество транзакций данного пользователя, в соответстии с параметрами объекта Pageable.\\
  \hline
  	\multirow{4}{*}{WalletServiceImpl} 
  	& delete(id: Long, userId: Long): Unit?
  	& Удаляет счёт по переданным признакам и удаляет все связанные с ним транзакции, ничего не возвращает, 
  	если операция была проведена успешно (Unit), возвращает null, если такой сущности не существует. 
  	Если произошла ошибка в процессе удаления, то бросает \mbox{ServiceException}.\\
  \cline{2-3}
  	& softDelete(id: Long, \\newId: Long,\\userId: Long): Unit?
  	& Удаляет счёт по переданным признакам и перемещает все связанные с ним транзакции на счёт с newId. 
  	Счёт с newId должен содержать ту же валюту, что и удаляемый счёт. Ничего не возвращает, если операция 
  	была проведена успешно (Unit), возвращает null, если такой сущности не существует. Если произошла 
  	ошибка в процессе удаления, то бросает \mbox{ServiceException}.\\
  \cline{2-3}
  	& calculateCostForecast(\\userId: Long)\\: CostForecast?
  	& Вычисляет прогнозируемые расходы для пользователя на основании информации о его зарплате, 
  	также вычисляет ежедневную сумму денег до зарплаты и количество дней до неё.\\
  \hline
  	\multirow{5}{*}{UserServiceImpl}
  	& register(registrationDetails\\: RegistrationDetails)
  	& Регистрирует нового пользователя в приложении. Бросает \mbox{ServiceBadRequestException} 
  	если пользователь с таким переданным email уже существует, пароль и подтверждение пароля не совпадают 
  	или если вылюты зарплаты не существует в базе.\\
  \cline{2-3}
  	& logout(principal\\: OAuth2Authentication)
  	& Удаляет из базы данных токен доступа OAuth 2.0 для данного пользователя, делая тем самым выход из приложения.\\
  \cline{2-3}
  	& changePassword(\\changeDetails: \\PasswordChangeDetails, \\userId: Long, \\principal: \\OAuth2Authentication)
  	& Заменяет текщий пароль пользователя на новый. Бросает \mbox{ServiceBadRequestException} 
  	если новый пароль не совпадает с подтверждением пароля, либо если старый пароль неверен.\\
  \cline{2-3}
  	& findSalaryInfo(\\userId: Long)\\: SalaryInfo?
  	& Возвращает информацию о зарплате для текущего пользователя.\\
  \cline{2-3}
  	& setSalaryInfo(salaryInfo\\: SalaryInfo,\\userId: Long)\\: SalaryInfo?
  	& Обновляет информацию о зарплате для текущего пользователя, бросает \mbox{ServiceBadRequestException}, 
  	если переданой валюты не существует в базе.\\
  \end{longtable}
\end{center}

\begin{center}
\begin{longtable}{ 
      | >{\centering}m{0.28\textwidth} 
      | >{\centering}m{0.31\textwidth} 
      | >{\centering\arraybackslash}m{0.35\textwidth}|}
  \caption{Классы и методы блока обработки ошибок}
  \label{table:exception_reference}\\
  \hline Класс & Метод  & Описание\\ \hline
  \endfirsthead
  \caption*{Продолжение таблицы \ref{table:exception_reference}}\\
  \hline 
    Класс & Метод & Описание\\
  \hline
  \endhead
  \hline
  \endfoot
  \hline
  \hline
  	WhiteLabelError-\\PageController
  	& error(request: \\HttpServletRequest): Any
  	& Представляет собой глобальный обработчик неопознанных ошибок, включая ошибки неверного URL. 
  	Возвращает в зависимости от MIME-типа запроса либо JSON-объект с инфрмацией об ошибке, либо html-страницу с ошибкой.\\
  \hline
    \multirow{2}{*}{RestExceptionHandler}
  	& handleMethodArgument-NotValid(
  			ex: MethodArgument-NotValidException,
            \\headers: HttpHeaders,
            \\status: HttpStatus,
            \\request: WebRequest
    \\): ResponseEntity<Any>
  	& Основной обработчик ошибок валидации при первичной обработке запроса, возвращает на клиент стандартизованный JSON-объект
  	со списком ошибок валидации и HTTP-кодом 400.\\
  \cline{2-3}
  	& handleHttpMessage-NotReadable(
            ex: HttpMessageNotReadable-Exception,
            headers: HttpHeaders,
            status: HttpStatus,
            request: WebRequest): ResponseEntity<Any>
  	& Обрабатывает ошибки десереализации из JSON в Java-объекты.\\
  \cline{2-3}
  	& handleTypeMismatch(
            ex: TypeMismatchException,
            headers: HttpHeaders,
            status: HttpStatus,
            request: WebRequest): ResponseEntity<Any>
  	& Обрабатывает ошибки типов в передаваемых с клиента параметрах запроса.\\
  \cline{2-3}
  	& handleConstraint-ViolationException(ex: Constraint-ViolationException): ResponseEntity<Any>
  	& Обрабатывает ошибки нарушения целостности базы данных, таких как попытка вставки неуникального 
  	значения в уникальное поле и др.\\
  \cline{2-3}
  	& handleBadRequest(ex: RequestException): ResponseEntity<Any>
  	& Обрабатывает обрабатывает ошибки нарушения бизнес логики приложения, такие как остутствие какой-либо сущности
  	в базе данных или невалидных вложенных сущностях. \\
  \cline{2-3}
  	& handleInternalError(ex: ControllerException): ResponseEntity<Any>
  	& Основной обработчик внутренних ошибок сервера, возвращает HTTP-код 500 Internal Server Error. \\
  \cline{2-3}
  	& handleInternalError-Explicit(ex: Exception): ResponseEntity<Any>
  	& Обработчик непредвиденных внутренних ошибок сервера, которые могут возникнуть в процессе эксплуатации системы, 
  	возвращает HTTP-код 500 Internal Server Error. \\
  \end{longtable}
\end{center}