\section{Анализ предметной области}

В данном разделе произведен краткий обзор существующих аналогов приложения;
сформулированы требования к разрабатываемому программному средству.

\subsection{Обзор аналогов}

В результате анализа предметной области было выявлено большое количество приложений
по личному учёту доходов и расходов.
В данном подразделе приведены три ярких представителя.

Основными критериями сравнения являются:
\begin{itemize}
\item пользовательский интерфейс;
\item удобство и быстрота использования;
\item возможность категоризации транзакций;
\item работа со счетами;
\item возможность просмотра статистики;
\item отображение текущего состояния счетов.
\end{itemize}

Классическим приложением по учёту доходов и расходов является система
<<Семейный бюджет>>, расположенная по адресу https://koshelek.org.
Данная система предоставляет широкую функциональность по контролю личного бюджета.
Предоставляет большие возможности по генерации отчётов, планированию будущих расходов.
Однако в следствии широких возможностей системы пострадал пользовательский интерфейс.
Внешний вид сайта устарел (рисунок \ref{fig:koshelek}), что также сказывается и на
удобстве использования. Также сайт перегружен множеством вложенных меню и мелких,
неочевидных иконок, что затрудняет работу с ним неподготовленному пользователю.

\begin{figure}[p]
\centering
\includegraphics[scale=0.55]{koshelek.png}
\caption{Раздел <<Доходы>> сайта системы <<Семейный бюджет>>}
\label{fig:koshelek}
\end{figure}

Кроме этого существует система под названием <<Drebedengi>> расположенная по адресу
http://drebedengi.org (рисунок \ref{fig:drebedengi}). Данная система позволяет
вести учёт доходов и расходов, перемещений между счетами, планировать бюджет
и контролировать текущее состояние счетов. Также приложение позволяет
контролировать долги. Пользовательский интерфейс более приятный, по сравнению с
<<Семейным бюджетом>>. Однако всё равно присутствует сложность восприятия из-за
большого количества чисел.

\begin{figure}[p]
\centering
\includegraphics[scale = 0.65]{drebedengi.png}
\caption{Главная страница пользователя сайта системы <<Drebedengi>>}
\label{fig:drebedengi}
\end{figure}

Третий аналог, отличающийся от описанных выше отсутствием перегруженного интерфейса
--- система <<Zenmoney>> (http://zenmoney.ru). Данное приложение позволяет работать
с личными счетами, категориями, транзакциями. Предоставляет возможности генерировать
отчёты, планировать бюджет, просматривать прогноз баланса. Главным недостатком данного \\
приложения является внешний вид приложение --- большинство элементов интерфейса
предоставлены без какой-либо стилизации (рисунок \ref{fig:zenmoney}).

Результат сравнения имеющихся аналогов с разрабатываемым приложением приведён в
таблице \ref{table:functions_comparing}.

При составлении таблицы учитывался
весь запланированный функционал, реализация некоторых функций возможна в
версиях, которые будут разработаны вне данного курсового проекта.


\begin{figure}[p]
\centering
\includegraphics[scale = 0.65]{zenmoney.png}
\caption{Страница транзакций сайта системы <<Zenmoney>>}
\label{fig:zenmoney}
\end{figure}


\begin{table}[p] \caption{Сравнение приведеных аналогов с
разрабатываемым приложением}
\label{table:functions_comparing}
\centering
\begin{tabular}{ | >{\centering}m{0.25\textwidth}
| >{\centering}m{0.07\textwidth}
| >{\centering}m{0.19\textwidth}
| >{\centering}m{0.19\textwidth}
| >{\centering\arraybackslash}m{0.155\textwidth}|}
\hline Функция & Kwit & Koshelek.org & Drebedengi.ru & Zenmoney.ru\\
\hline Современный интерфейс & + & - & + & -\\
\hline Простота в использовании & + & - & - & +\\
\hline Мультивалютность & + & + & + & +\\
\hline Простота в использовании & + & - & - & +\\
\hline Работа со счетами & + & + & + & +\\
\hline Работа с категориями & + & + & + & +\\
\hline Возможность \\ генерировать отчёты & - & + & + & +\\
\hline Прогнозирование & + & + & + & +\\
\hline
\end{tabular}
\end{table}

\subsection{Постановка задачи}

Целью данного курсового проекта является разработка:
\begin{itemize}
\item добавления, удаления и изменения транзакций;
\item добавления, удаления и изменения категорий;
\item добавления, удаления и изменения счетов;
\item возможности удаления категорий и счетов с переносом всех \\ транзакций на другой счёт;
\item возможности подсчёта статистики по категориям за \\ произвольный период времени;
\item подсчёта ежедневной суммы до зарплаты, прогноза будущих затрат \\ за последнее время.
\end{itemize}

Программное средство должно представлять собой сервер c REST API и веб-клиент.
Данное сочетание позволит в дальнейшем развить данное приложение в полноценную
кроссплатформенную систему.