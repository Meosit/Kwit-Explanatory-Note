\sectioncentered*{Введение}
\addcontentsline{toc}{section}{Введение}

Времена, когда компьютеры и интернет использовались лишь как инструменты для
сложных научных вычислений, давно прошли. Сейчас информационные
технологии помогают нам работать, отдыхать, делать покупки, обмениваться
информацией и общаться с людьми в любой точке земного шара. 

Не малую роль в обеспечении ширины спектра задач, в решении которых участвуют информационные технологии, 
играет Интернет. APRANET, созданный в 1969 году как научная сеть для связи
исследовательских центров, относительно быстро развился в то, что мы привыкли
называть интернетом, так как с ростом количества информации  у
человечества необходимость в быстром и простом способе обмена ею увеличивалась.

Благодаря доступности интернета, в наши дни он широко используется для общения.
Появившиеся сравнительно недавно средства мгновенного обмена сообщениями, а
позже и социальные сети, они быстро набрали свою аудиторию и превратились в
мощные и многофункциональные платформы для общения в интернете.

На территории СНГ наибольшей популярностью обладает социальная сеть
\vk{}. Данная социальная сеть отличается широким, но в то же время не
перегруженным функционалом, удобным и приятным дизайном. Несмотря на
большое количество функций, наиболее популярной возможностью \vk{} является
обмен сообщениями. Несмотря на насыщенность рынка программного обеспечения
различными мессенджерами, \vk{} отличает охват широкой аудитории. Это снимает
необходимость подстраиваться под разных людей с разными программами для
мгновенного обмена сообщениями и позволяет вести всю неделовую \mbox{переписку}
в одном месте.

Однако специфика \vk{}, связанная с его функционалом браузерной социальной
сети: необходимость запуска браузера, слабая поддержка уведомлений на рабочем
столе, отвлекающие элементы в веб-интерфейсе и некоторые другие проблемы, не
позволяет добиться максимального удобства использования в качестве системы мгновенного обмена сообщениями.

Целью данного курсового проекта является разработка кроссплатформенного клиента
для \vk{}, предоставляющего удобный доступ ко всему функционалу, связанному с
обменом сообщениями.
