\sectioncentered*{Введение}
\addcontentsline{toc}{section}{Введение}

В наше время существует невероятное количество разных способов траты денег, некоторые 
из них являются необходимостью, например оплата коммунальных услуг или телефонной 
связи, а некоторые простым развлечением, начиная от большого количества различных 
съедобных вкусностей и заканчивая удивительными способами экстремального отдыха.

Появилась проблема контроля за расходами, учётом целей трат и их объемов. Для 
решения данной проблемы с течением времени использовались разные подходы, 
такие как простая запись раходов и доходов в тетрадь или ведение таблиц Excel.
С развитием технологий разработки мобильных и веб-приложений стали появляться отдельные 
средства для полного контроля над личными финансами.

Широкое применение в области учёта финансов нашли аналитические технологии для прогнозирования
и анализа данных, прогнозирование на основе эконометрических, регрессионных и нейросетевых
алгоритмах.

Целью данного курсового проекта является разработка веб-приложения для простого и удобного 
ведения учёта личных доходов и расходов с внедрением возможностей оценки текущего баланса
и прогнозированием будущих раходов. Внимание планируется сконцентрировать на простоту 
использования и прослеживания текущего состояния счетов.