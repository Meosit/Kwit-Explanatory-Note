\sectioncentered*{Введение}
\addcontentsline{toc}{section}{Введение}

За 2016 год потребительские расходы домашних хозяйств Беларуси по категории
<<прочие товары и услуги>> составляют 8.7\% от общего дохода [1]. С 2012 года данный показатель
вырос на один процент. Такие, казалось бы, небольшие числа говорят о том, что с каждым годом
контроль над расходами белорусов постепенно снижается.

Среди общества присутствует проблема контроля за расходами, учётом целей трат и их
объемов. Для решения проблемы потери контроля за расходами с течением времени
использовались разные подходы, такие как простая запись расходов и доходов в тетрадь или
ведение таблиц Excel. С развитием технологий разработки мобильных и веб-приложений стали
появляться отдельные средства для полного контроля над личными финансами.

Широкое применение в области учёта финансов нашли аналитические технологии для прогнозирования
и анализа данных, прогнозирование на основе эконометрических, регрессионных и нейросетевых
алгоритмах.

Целью данного курсового проекта является разработка веб-приложения для простого и удобного
ведения учёта личных доходов и расходов с внедрением возможностей оценки текущего баланса
и прогнозированием будущих расходов. Внимание планируется сконцентрировать на простоту
использования и прослеживания текущего состояния счетов.

Задачи, которые предполагается решить в рамках курсового проекта:
\begin{itemize}
\item изучение и обобщение знаний о языке программирования \kt{};
\item изучение архитектуры REST и применение её в разработке API;
\item изучение различных алгоритмов прогнозирования временных \mbox{рядов};
\item изучение и использование программной платформы AngularJS.
\end{itemize}