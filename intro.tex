\sectioncentered*{Введение}
\addcontentsline{toc}{section}{Введение}

За 2016 год потребительские расходы домашних хозяйств Беларуси по категории 
<<прочие товары и услуги>> составляет 8.7\% от общего дохода [1]. С 2012 года данный показатель
вырос на один процент. Такие, казалось бы, небольшие числа говорят о том, что с каждым годом
контроль над расходами белорусов постепенно снижается.

Среди общества присутствует проблема контроля за расходами, учётом целей трат и их 
объемов. Для решения данной проблемы с течением времени использовались разные подходы, 
такие как простая запись раходов и доходов в тетрадь или ведение таблиц Excel.
С развитием технологий разработки мобильных и веб-приложений стали появляться отдельные 
средства для полного контроля над личными финансами.

Широкое применение в области учёта финансов нашли аналитические технологии для прогнозирования
и анализа данных, прогнозирование на основе эконометрических, регрессионных и нейросетевых
алгоритмах.

Целью данного курсового проекта является разработка веб-приложения для простого и удобного 
ведения учёта личных доходов и расходов с внедрением возможностей оценки текущего баланса
и прогнозированием будущих раходов. Внимание планируется сконцентрировать на простоту 
использования и прослеживания текущего состояния счетов.

Задачи, которые предполагается решить в рамках курсового проекта:
\begin{itemize}
	\item изучение и обобщение знаний о языке программирования \kt{};
	\item изучение арихитектуры REST и примение её в расзработке API;
	\item изучение различных алгоритмов прогнозирования временных \mbox{рядов};
	\item изучение и использование программной платформа AngularJS.
\end{itemize}