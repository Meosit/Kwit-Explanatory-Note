\sectioncentered*{Заключение}
\addcontentsline{toc}{section}{Заключение}

В рамках данной курсовой работы было создано пррограммное средство \vkapp{},
предназначенное для мгновенного обмена сообщениями через социальную сеть \vk{}.
Исходный код программного средства без изменений может быть скомпилирован
под такие операционные системы, как Windows, Mac OS и большинство дистрибутивов
Linux.

Модули для кэширования, работой с базой данных и взаимодействия с API \vk{} были
полностью разработаны Константином Тереховым. Классы, предназначенные для
осуществления сетевых запросов, внутрипрограммного представления объектов
социальной сети, виджеты для отображения сообщений и диалогов, логика
графического интерфейса --- Иваном Шимко и Владиславом Савёнком. Проектирование
общей архитектуры приложения и периодическая инспекция кода осуществлялись
совместно.

В процессе разработке были приобретены навыки по работе с сетью, расширены
знания фреймворка Qt, языка программирования C++, паттернов
объектно-ориентированного проектирования и концепции \mvc{}.
Разработка модуля для сохранения кэша потребовала дополнительного изучения
реляционных баз данных и языка SQL. При разработке модуля для взаимодействия с
\vk{} был подробно изучен формат обмена данными JSON, используемый API
социальной сети. Коллективная разработка данного программного средства позволила
приобрести опыт работы с системой контроля версий git.

В связи со сложностью и объёмностью данного проекта, на что косвенно указывает
практически полное отсутствие неофициальных клиентов для данной социальной сети
для десктопных ОС, не весь изначально запланированный
функционал был реализован в полной мере.

Программное средство имеет большой потенциал для развития и совершенствования,
однако текущая версия содержит необходимый каркас, который обеспечивает
достаточную гибкость для расширения поддержки функций социальной сети \vk{}.

В будущих версиях планируется расширить перечень отображаемых вложений в личные
сообщения, усовершенствовать поддержку групповых диалогов, добавить настройки пометки
сообщений прочитанными и обеспечить возможность настройки уведомлений о
новых сообщениях. Также возможно добавление встроенного плеера для
воспроизведения прикрепленных к личным сообщениям аудиозаписей и взаимодействия
с хранилищем аудиозаписей \vk{}.
